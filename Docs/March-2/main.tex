% Copyright 2004 by Till Tantau <tantau@users.sourceforge.net>.
%
% In principle, this file can be redistributed and/or modified under
% the terms of the GNU Public License, version 2.
%
% However, this file is supposed to be a template to be modified
% for your own needs. For this reason, if you use this file as a
% template and not specifically distribute it as part of a another
% package/program, I grant the extra permission to freely copy and
% modify this file as you see fit and even to delete this copyright
% notice. 

\documentclass{beamer}
\usepackage{amsmath}
\usepackage{amssymb}
\usepackage{graphicx}
\graphicspath{ {Images/} }
\usepackage{graphicx}


\usepackage{tikz}
\usetikzlibrary{shapes.geometric, arrows}
\tikzstyle{startstop} = [rectangle, rounded corners, minimum width=3cm, minimum height=1cm,text centered, draw=black, fill=red!30]
\tikzstyle{io} = [trapezium, trapezium left angle=70, trapezium right angle=110, minimum width=3cm, minimum height=1cm, text centered, draw=black, fill=blue!30]
\tikzstyle{process} = [rectangle, minimum width=3cm, minimum height=1cm, text centered, draw=black, fill=orange!30]
\tikzstyle{decision} = [diamond, minimum width=3cm, minimum height=1cm, text centered, draw=black, fill=green!30]
\tikzstyle{arrow} = [thick,->,>=stealth]
\tikzstyle{Memory} = [cylinder,shape border rotate=90,draw,minimum height=2.5cm,minimum width=2cm]

% There are many different themes available for Beamer. A comprehensive
% list with examples is given here:
% http://deic.uab.es/~iblanes/beamer_gallery/index_by_theme.html
% You can uncomment the themes below if you would like to use a different
% one:
%\usetheme{AnnArbor}
%\usetheme{Antibes}
%\usetheme{Bergen}
%\usetheme{Berkeley}
%\usetheme{Berlin}
%\usetheme{Boadilla}
%\usetheme{boxes}
%\usetheme{CambridgeUS}
%\usetheme{Copenhagen}
%\usetheme{Darmstadt}
%\usetheme{default}
%\usetheme{Frankfurt}
%\usetheme{Goettingen}
%\usetheme{Hannover}
%\usetheme{Ilmenau}
%\usetheme{JuanLesPins}
%\usetheme{Luebeck}
\usetheme{Madrid}
%\usetheme{Malmoe}
%\usetheme{Marburg}
%\usetheme{Montpellier}
%\usetheme{PaloAlto}
%\usetheme{Pittsburgh}
%\usetheme{Rochester}
%\usetheme{Singapore}
%\usetheme{Szeged}
%\usetheme{Warsaw}

%\title{AI- Project}

% A subtitle is optional and this may be deleted
\title{Formalising Euler Method in Isabelle}

\author{Rishabh Manoj}
% - Give the names in the same order as the appear in the paper.
% - Use the \inst{?} command only if the authors have different
%   affiliation.

\institute{International Institute of Information Technology - Bangalore}

\date{\today}
% - Either use conference name or its abbreviation.
% - Not really informative to the audience, more for people (including
%   yourself) who are reading the slides online

\subject{Theorem Proving}
% This is only inserted into the PDF information catalog. Can be left
% out. 

% If you have a file called "university-logo-filename.xxx", where xxx
% is a graphic format that can be processed by latex or pdflatex,
% resp., then you can add a logo as follows:

% \pgfdeclareimage[height=0.5cm]{university-logo}{university-logo-filename}
% \logo{\pgfuseimage{university-logo}}

% Delete this, if you do not want the table of contents to pop up at
% the beginning of each subsection:
\AtBeginSubsection[]
{
  \begin{frame}<beamer>{Outline}
    \tableofcontents[currentsection,currentsubsection]
  \end{frame}
}

% Let's get started
\begin{document}

\begin{frame}
  \titlepage
\end{frame}



% Section and subsections will appear in the presentation overview
% and table of contents.

\section{Formal Analysis of ODE}

\begin{frame}{Formal Analysis of ODE}{}
\begin{itemize}
    \item The paper \textit{Numerical Analysis of Ordinary Differential
Equations in Isabelle/HOL} aims to formalise initial value problems (IVPs) of ODEs and prove the Picard-Lindel\"of theorem.\\ 
    \item They use a generic one-step methods for numerical approximation of the solution and give an executable specification of the Euler method as an instance
of one-step methods.  
\end{itemize}

\end{frame}


\section{Picard-Lindel\"of theorem}

\begin{frame}{Picard-Lindel\"of theorem}{}

  Given an ODE of the form 
   $$ \frac{du}{dt} = f(t,u(t)), \ \ u(t_0) = x_0$$

 if $f$ is \textit{Lipschitz} continuous in $u$ and continuous in $t$ then $\frac{du}{dt}$ has a unique solution.
 
 A fn. is called \textit{Lipschitz} if it's derivative is bounded. Intutively, the fn. is limited in how fast it can change. 
\end{frame}

\begin{frame}{Proof}{}

  Integrating the ODE gives 
   $$ u(t) - u(t_0)= \int_{t_0}^{t} f(s,u(s)) ds$$
   
   Set
      $$\phi_0(t) = y_0$$
    and
    $$\phi_{k+1}(t) = y_0 + \int_{t_0}^{t} f(s,\phi_{k}(s)) ds$$    

 Using Banach fixed point theorem, it can be shown that the sequence $\phi_k$ is \emph{convergent} and the limit is the solution. \cite{Wiki}
\end{frame}


\section{One Step Method}

\begin{frame}{One-step Method}{}
One-step method approximates a function (the solution)
in discrete steps, each step operating exclusively on the results of one previous
step. For one-step methods in general, one can give assumptions under which
the method works correctly \cite{ODE}
\begin{itemize}
      \item \textbf{Consistency}:- If the error in one step
goes to zero with the step size, the one-step method is called consistent\\
    \item \textbf{Convergence}:- The global error, the error after a series of steps, goes to zero with the step size\\
    \item \textbf{Stability}:- For efficiency reasons, we want to limit the precision of our calculations which causes rounding errors.The error between the
ideal and the perturbed function goes to zero with the step size.\\
  \end{itemize}

\end{frame}


\begin{frame}{One-step Method}
\begin{itemize}
\item It can be proved that a consistent one-step methods are convergent and stable.For detailed proof refer \cite{ODE}
\item We show that Euler method is consistent and can therefore be used to approximate IVPs.
\item We show that the Euler method is convergent which proves the Picard-Lindel\"of theorem
\end{itemize}
\end{frame}


\begin{frame}{One-step Method}{Consistency}
  \begin{center}
            \begin{equation*}
            \begin{aligned}
                u'(t) &= f(t,u(t))    \\
                gf_j &\approx u(t_j) \\
                gf_{j+1} &= L(gf_j,h,t_j) \\
                L(gf_j,h,t_j) &= gf_j + h*f(t_j,gf_j) \ \ (h = t_{j+1} - t_j)\\ 
                \text{Local Error} &= ||u(t_{j+1}) - L(gf_j,h,t_j)||
            \end{aligned}
            \end{equation*}
            \end{center}
        
        A one-step method is considered consistent with $u$ of order $p$ if the local error is in  $\mathcal{O}(h^{p+1})$ (i.e)   $\rightarrow \raisebox{.5pt}
{\textcircled{\raisebox{-.95pt} {1}}}$ \cite{Wolf}
        $$||u(t_{j+1}) - L(gf_j,h,t_j)|| \leq B.h^{p+1}$$ 
            \hfill where $B$ is a constant 
  
  
\end{frame}

\begin{frame}{Euler Method}{Consistency}
  
  Using the Taylor series expansion
 
 $$f(x) = f(a) + (x-a)*f'(a) + \frac{(x-a)^2}{2}*f''(a) + \cdots$$
 
 $$\implies f(x) - (f(a) + (x-a)*f'(a)) = \mathcal{O}(h^2)$$

Euler Method:                              
    $$u(t+h) = u(t) + h*(u'(t))$$                                        
   $\implies$ The Local Error of Euler Method is in $\mathcal{O}(h^2)$.\\
   
   From $\raisebox{.5pt}
{\textcircled{\raisebox{-.95pt} {1}}}$ we say that,\\ \hfill Euler Method is consistent and therefore convergent and stable                         
\end{frame}

\begin{frame}{References}
\bibliographystyle{plain}
\bibliography{citations}
\end{frame}

\end{document}



